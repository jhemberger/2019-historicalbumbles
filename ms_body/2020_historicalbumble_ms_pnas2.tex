\documentclass[11pt,]{article}
\usepackage[]{mathpazo}
\usepackage{amssymb,amsmath}
\usepackage{ifxetex,ifluatex}
\usepackage{fixltx2e} % provides \textsubscript
\ifnum 0\ifxetex 1\fi\ifluatex 1\fi=0 % if pdftex
  \usepackage[T1]{fontenc}
  \usepackage[utf8]{inputenc}
\else % if luatex or xelatex
  \ifxetex
    \usepackage{mathspec}
  \else
    \usepackage{fontspec}
  \fi
  \defaultfontfeatures{Ligatures=TeX,Scale=MatchLowercase}
\fi
% use upquote if available, for straight quotes in verbatim environments
\IfFileExists{upquote.sty}{\usepackage{upquote}}{}
% use microtype if available
\IfFileExists{microtype.sty}{%
\usepackage{microtype}
\UseMicrotypeSet[protrusion]{basicmath} % disable protrusion for tt fonts
}{}
\usepackage[margin = 1in]{geometry}
\usepackage{hyperref}
\hypersetup{unicode=true,
            pdftitle={The fate of bumble bees in the north-central United States is linked to historical patterns of agricultural intensification},
            pdfauthor={Jeremy Hemberger; Michael Crossley; Claudio Gratton},
            pdfkeywords={Agricultural intensification, Bumble bees, bee decline, agroecosystems},
            pdfborder={0 0 0},
            breaklinks=true}
\urlstyle{same}  % don't use monospace font for urls
\usepackage{graphicx,grffile}
\makeatletter
\def\maxwidth{\ifdim\Gin@nat@width>\linewidth\linewidth\else\Gin@nat@width\fi}
\def\maxheight{\ifdim\Gin@nat@height>\textheight\textheight\else\Gin@nat@height\fi}
\makeatother
% Scale images if necessary, so that they will not overflow the page
% margins by default, and it is still possible to overwrite the defaults
% using explicit options in \includegraphics[width, height, ...]{}
\setkeys{Gin}{width=\maxwidth,height=\maxheight,keepaspectratio}
\IfFileExists{parskip.sty}{%
\usepackage{parskip}
}{% else
\setlength{\parindent}{0pt}
\setlength{\parskip}{6pt plus 2pt minus 1pt}
}
\setlength{\emergencystretch}{3em}  % prevent overfull lines
\providecommand{\tightlist}{%
  \setlength{\itemsep}{0pt}\setlength{\parskip}{0pt}}
\setcounter{secnumdepth}{0}
% Redefines (sub)paragraphs to behave more like sections
\ifx\paragraph\undefined\else
\let\oldparagraph\paragraph
\renewcommand{\paragraph}[1]{\oldparagraph{#1}\mbox{}}
\fi
\ifx\subparagraph\undefined\else
\let\oldsubparagraph\subparagraph
\renewcommand{\subparagraph}[1]{\oldsubparagraph{#1}\mbox{}}
\fi

%%% Use protect on footnotes to avoid problems with footnotes in titles
\let\rmarkdownfootnote\footnote%
\def\footnote{\protect\rmarkdownfootnote}

%%% Change title format to be more compact
\usepackage{titling}

% Create subtitle command for use in maketitle
\providecommand{\subtitle}[1]{
  \posttitle{
    \begin{center}\large#1\end{center}
    }
}

\setlength{\droptitle}{-2em}

  \title{\textbf{The fate of bumble bees in the north-central United States is
linked to historical patterns of agricultural intensification}}
    \pretitle{\vspace{\droptitle}\centering\huge}
  \posttitle{\par}
    \author{Jeremy Hemberger \\ Michael Crossley \\ Claudio Gratton}
    \preauthor{\centering\large\emph}
  \postauthor{\par}
      \predate{\centering\large\emph}
  \postdate{\par}
    \date{February 25, 2020}

\usepackage{setspace}

\doublespacing
\usepackage[left]{lineno}
\linenumbers
\usepackage{dcolumn}
\usepackage{caption}
\usepackage{float}
\usepackage{afterpage}
\usepackage{siunitx}

\begin{document}
\maketitle
\begin{abstract}
The decline of North American bumble bees has been tentatively linked to
changes in agricultural practices over the last century. Researchers
posit that a switch from the diverse and less intensive agricultural
systems of the early 1900's to the largely monocultural, heavy-input
systems of today has impacted both nesting habitat and floral resource
availability with negative consequences for bumble bee populations and
communities. Despite several long-term analyses highlighting concerning
population trends, no studies have yet explicitly linked metrics of
agricultural intensification to bumble bee declines. Using an extensive
long term set of both bumble bee records and US agricultural census
records, we show that metrics of increasing agricultural intensity,
specifically proportion cropland and a decrease in crop richness, are
clearly associated with the decline of nearly half of bumble bee species
in the north-central US. Additionally, we document positive species
associations with crop richness and specific metrics such as proportion
pasture even under areas of intensive cultivation, suggesting that
bumble bee declines are linked to not just the extent of agriculture,
but rather the practices implemented on the ground. In addition to
spurring future avenues for controlled, manipulative experiments, our
results suggest that changes to our agricultural practice and policies
are required in order to limit additional declines of Midwestern bumble
bees.
\end{abstract}

\captionsetup[table]{labelformat=empty}

\hypertarget{introduction}{%
\section{Introduction}\label{introduction}}

Extensive range contractions (Kerr et al. 2015) and possible local
extinctions (Rasmont and Iserbyt 2012) of a number of bee species has
sounded an alarm among conservationists across the world
(Steffan-Dewenter et al. 2005; Tylianakis 2013). Of bee population
trends that have been well documented, bumble bees (Apidae: Bombus) are
perhaps the most well-studied with reports of many species declining
across Europe and North/South America (Biesmeijer 2006; Colla and Packer
2008; Grixti et al. 2009; Cameron et al. 2011; Bartomeus et al. 2013;
Morales et al. 2013; Wood et al. 2019). Owing to their importance as
pollinators in both commercial crops (Klein et al. 2007) and natural
systems (Ollerton, Winfree, and Tarrant 2011), understanding the factors
driving bumble bee decline is essential for their conservation and
management.

A number of studies have posited mechanisms underpinning bumble bee
declines, including climate change (Kerr et al. 2015; Marshall et al.
2017; Sirois-Delisle and Kerr 2018), habitat loss (Grixti et al. 2009;
Goulson et al. 2015; Soroye, Newbold, and Kerr 2020), the proliferation
of parasites and disease (Cameron et al. 2011, @McArt2017), and
competition with managed bees (Torné-Noguera et al. 2016; Cane and
Tepedino 2017; Mallinger, Gaines-Day, and Gratton 2017; Ropars et al.
2019). Among potential drivers the expansion and intensification of
agriculture is consistently cited as a key factor in long-term declines
of once common species in both the US (Grixti et al. 2009; Colla and
Packer 2008; Jacobson et al. 2018) and Europe (Goulson et al. 2005;
Carvell et al. 2006). Intensification here refers to a suite of
attributes associated with agricultural systems that depend on expansive
(large scale) monocultures associated with high inputs of agrichemicals
(pesticides, synthetic fertilizers) and disturbances such as annual
tillage practices (lots of refs). Intensification of agriculture
therefore includes both attributes within a farming system (i.e.,
pesticide use, crop types used), and the effects on agricultural
landscapes (i.e., decreased diversity of crops, amount of land in
cultivation). A switch from the more diverse and less intensive
agricultural production systems of the early 1900's to intensive
production systems practiced in many developed countries may have
contributed to bumble bee declines due to the loss of preferred host
plant species (Wood et al. 2019) and nesting/overwintering habitat
(Liczner and Colla 2019).

In the United States, analysis of museum records suggests that bumble
bee declines began during periods of agricultural intensification in the
mid 20th century (Grixti et al. 2009). Despite several long-term
analyses highlighting concerning population trends as well as studies
describing the importance of historical land-use (Cusser, Neff, and Jha
2018), there has been no explicit exploration of whether increased
agricultural intensification is related to bumble bee declines. One
reason for this is the paucity of long-term data of both bumble bee
occurrence and historical agricultural patterns at large spatial scales,
as well as the difficulty in assessing potential drivers experimentally.
Additionally, variable temporal patterns of species declines have been
documented (e.g., long-term declines of European species (Williams,
Araújo, and Rasmont 2007) vs.~the recent crash of \emph{Bombus affinis}
Cresson in North America (Giles and Ascher 2006), suggesting that the
mechanisms driving declines may be species- and/or region-specific.
Moreover, some species populations appear stable, or may be increasing
in both abundance and range (e.g., \emph{B. impatiens} Cresson:
Richardson et al. 2018; Wood et al. 2019). In the US, increases and
decreases in bumble bee populations appear to be temporally linked,
suggesting that there are shared drivers leading some species to thrive
while others struggle.

To understand the drivers of bumble bee declines, several approaches
have been developed to overcome data limitations, including experiments
leveraging space-for-time substitution, as well as curation of
historical museum collections to understand species trends (Bartomeus et
al. 2019). While unable to draw specific causal connections, the use of
historical data provides a unique perspective on large-scale temporal
trends that experimental approaches cannot provide. Several statistical
approaches have been developed to account for the known biases of museum
collection records, allowing new data sources to be implemented and used
to explore bumble bee population trends (Pearce and Boyce 2006;
Bartomeus et al. 2013, 2019). The continued addition of museum
collection records to repositories such as the Global Biodiversity
Information Facility (GBIF) combined with extensive, modern surveys of
bumble bee fauna (e.g., Bumble Bee Watch) offer a wealth of
species-specific spatial distribution patterns. Moreover, data
describing detailed agricultural production trends in the United States
(Crossley et al.~\emph{in review}) now offer a unique opportunity to
assess the relationship between agricultural patterns and bee population
trends over time.

To this end, we explore the relationships between long-term bumble bee
population trends and historical patterns of agricultural
intensification in the north-central US using an extensive data set of
historical bumble bee museum records, modern citizen-science surveys and
agronomic metrics distilled from United States Census of Agriculture
records over the period 1870-2018. Using relating species relative
abundance to agricultural intensity at the county level, we expected
that increasing levels agricultural intensification at the county level
as measured by proportion cropland, number of crops grown, proportion of
pasture, and proportion of land treated with insecticides, are
associated with: (1) a decrease in the relative abundance of species of
conservation concern; and (2) an increase in the relative abundance of
species known to be stable.

\hypertarget{methods}{%
\section{Methods}\label{methods}}

We focused our study on the North Central US states of Minnesota,
Wisconsin, Iowa, Illinois, Michigan, and Indiana as these states share a
similar biogeographic contexts and agricultural history. We limited our
analysis to bumble bee species whose core ranges overlapped these states
and to limit comparisons of species at their range limits, including:
\emph{B. affinis} Cresson, 1863; \emph{B. impatiens} Cresson, 1863;
\emph{B. griseocollis} DeGeer, 1773; \emph{B. bimaculatus} Cresson,
1863; \emph{B. auricomus} Robertson, 1903; \emph{B. ternarius} Say,
1837; \emph{B. vagans} Smith, 1854; \emph{B. borealis} Kirby, 1837;
\emph{B. citrinus} Smith, 1854; \emph{B. pensylvanicus} De Geer, 1773;
\emph{B. fervidus} Fabricius, 1798; \emph{B. rufocinctus} Cresson, 1863;
and \emph{B. terricola} Kirby, 1837. Additionally, several species of
conservation concern known to be in decline nationally (including
\emph{B. affinis}, \emph{B. terricola} and \emph{B. pensylvanicus}) have
core historic ranges in the north-central US.

\hypertarget{bumble-bee-record-data}{%
\subsubsection{Bumble bee record data}\label{bumble-bee-record-data}}

We obtained bumble bee records using the Global Biodiversity Information
Facility (GBIF), querying for all records within our study region. These
data were combined with records from the North American Bumble Bee Watch
program (www.bumblebeewatch.org) provided by the Xerces Society for
Invertebrate Conservation. We then filtered records to include only
species known to occur within our study region and which were
appropriately geo-referenced (i.e., having associated longitude and
latitude). Each record was assigned to a county based on its collection
coordinates so that they could be matched to county-level agricultural
census data. Because 95\% of records were from 1890 and beyond, we are
confident that our county assignments are accurate as changes in county
geographic extent in this region were largely complete by 1890
{[}Crossley et al.~\emph{in review}{]}.

Temporal comparisons of museum and incidental records can be problematic
due to collector/spatial bias and non-standardized collection techniques
(Bartomeus et al. 2013, 2019; Richardson et al. 2018). To account for
this, we analyzed records using a variety of techniques to control for
potential biases. We filtered the dataset to include only single
individual `sampling events' (unique combination of species, date,
location, and collector), following Richardson et al. (2018). All
analyses presented below were conducted using both the full and reduced
datasets. As we found no difference in the analyses between the full and
filtered (Appendix 1), we present results from the full dataset.

\hypertarget{calculating-temporal-patterns-of-abundance-and-diversity}{%
\subsubsection{Calculating temporal patterns of abundance and
diversity}\label{calculating-temporal-patterns-of-abundance-and-diversity}}

We first examined temporal trends species diversity for our study region
independent of the agricultural census (Grixti et al. 2009; Bartomeus et
al. 2013; Richardson et al. 2018; Jacobson et al. 2018; Wood et al.
2019). In this analysis, we used the approach of Bartomeus et al. (2013)
and created temporal bins of records such that there were approximately
the same number of bee observations per temporal bin. Bins were created
using quantiles and the \texttt{rbin} package in R. We created several
binning strategies to determine if the number of bins affected our
results (Bartomeus et al. 2013), including a total of 3, 5, 8, and 15
temporal bins. Because trends were similar regardless of the number of
bins, we present results from the 8-bin analyses for changes in relative
abundance and results from the 15-bin analyses for estimation of species
richness over time.

In order to estimate species richness changes over time, we rarefied
records to generate estimates of mean species richness for each of 15
temporal bins with 95\% confidence intervals using the \texttt{iNEXT}
package (Hsieh, Ma, and Chao 2016). We then fit a linear model to
determine if there was a statistically clear change in species richness
over time. Because each time bin contained a different number of years,
we used the midpoint of each bin as the value from which to construct
the model. We also conducted a permutation test to determine a p-value
of the relationship as the assumption of normally distributed data for
such a small sample may be violated. Over 1,000 permutations, we
randomly shuffled the temporal bin order, calculating the correlation
between bin and species richness estimates each permutation, with the
p-value equalling the fraction of permuted correlations greater or less
than the true chronological correlation value.

\hypertarget{estimating-change-in-county-occupancy}{%
\subsubsection{Estimating change in county
occupancy}\label{estimating-change-in-county-occupancy}}

Changes in relative abundance may not fully capture declines if species
remain stable in occupied counties while the number of occupied counties
decreases over time. To account for changes in county-level occupancy
(i.e., a proxy for range), we modeled the number of occupied counties
per equal-record temporal bin for each study species using a generalized
linear model, predicting number of counties as a function of temporal
bin.

\hypertarget{historical-agriculture-data}{%
\subsubsection{Historical agriculture
data}\label{historical-agriculture-data}}

To assess the extent, diversity, and intensity of agriculture, we used
county-level agricultural census data projected and geographically
corrected by Crossley et al.~(in review). Our objective was to estimate
county-level (n=535 counties) agricultural metrics, which included
aspects of both extensive and intensive farming practices. Following
Crossley et al., we limited our analysis to the 18 most common crops
which together represent a majority of cropland area study region
(Crossley et al.~in review). For each county by census year, we
calculated crop richness and the proportion of county area in cropland
as two measures of agricultural intensity.

We also extracted additional aspects of farm management hypothesized to
be putative drivers of bumble bee declines. These included acreage of
specific agricultural land-uses and practices known to affect bumble
bees including pasture acreage and acreage treated with
arthropod-targeting insecticides. Because these two variables were not
available across the same temporal extent as crop richness and
proportion of county in cropland, we conducted a separate analysis from
1982 onward where pasture acreage and insecticide treated acreage were
available at the county level. For our analysis, we converted pasture
and insecticide treated acreage to proportion of county area to match
the scale of other model variables, namely proportion of county in
agriculture.

\hypertarget{pairing-bumble-bee-records-with-historical-agriculture-dataset}{%
\subsubsection{Pairing bumble bee records with historical agriculture
dataset}\label{pairing-bumble-bee-records-with-historical-agriculture-dataset}}

Because agricultural census data are collected every decade, not every
bumble bee record was collected in a year concurrently with a census. As
such, we paired records such that they were ± 5 years of the nearest
agricultural census date (e.g., bumble bee records from 1926-1935 were
paired with the 1930 agricultural census). While this pairing may not
perfectly reflect the state of agriculture experienced by collected
bumble bees, we posit that it is still a meaningful pairing given that a
difference of up to 5 years is not likely to manifest in large,
county-level changes in agricultural practices that we observed
occurring over several decades. To verify this assumption, we performed
analyses with a stricter up to ± 2-year pairing rule and found similar
results (Appendix 2), thus we present the analysis with the full dataset
and up to ± 5-year pairing.

\hypertarget{relating-changes-in-relative-abundance-to-agricultural-intensity}{%
\subsubsection{Relating changes in relative abundance to agricultural
intensity}\label{relating-changes-in-relative-abundance-to-agricultural-intensity}}

We constructed statistical models to examine whether changes in metrics
of agricultural intensification were related to bumble bee relative
abundance. For each bumble bee species, we fit a generalized linear
model with a binomial error structure to predict county-level relative
abundance (for a given agricultural census year, number of records of a
given species divided by total number of bumble bee records) as a
function of number of crops grown within a county, the proportion of
county area in agriculture and the year of the agricultural census data
associated with that bumble bee record, allowing us to determine a
temporal trend in species relative abundance while taking into account
agricultural intensification. Observations were weighted by the number
of bumble bee records in a county by agricultural census year -
effectively giving more weight to counties that had greater sampling
intensity. We assumed that these counties provided more accurate
estimates of species relative abundance at any given time. Additionally,
we further limited our analysis to only include counties by year
combinations with greater than 5 total bumble bee records to eliminate
the presence of counties with little sampling effort, and for which low
numbers may artificially inflate the relative abundance of given
species.

Because of the spatial nature of these data, we tested for spatial
autocorrelation in the residuals using a Moran's I test in the
\texttt{spdep} package (similar to Meehan et al. 2011; and Meehan and
Gratton 2015). We also checked for temporal autocorrelation within the
response of each species. Because neither spatial nor temporal
autocorrelation were found to be problematic, we utilized the
generalized linear model framework described above across all species.

To illustrate the estimated change in relative abundance and range, we
used our fitted models to predict occurrence across historical ranges of
each species within our study area using county-level agricultural
intensification metrics. These maps describe the predicted probability
of occurrence in each county as function of crop diversity and
proportion of a county in cropland over time. We selected 6 time points
for which to fit models: 1870, 1900, 1930, 1959, 1974, and 2012 to
depict how county-level agricultural suitability for bumble bees has
changed over time.

Another series of models were constructed to test more detailed
hypothesized drivers of bumble bee decline. Generalized linear models
with a binomial error structure were fit to predict county-level species
relative abundance as a function of pasture acreage, acreage treated
with arthropod-targeting insecticides, and the agricultural census year.
Because the agricultural census did not begin capturing insecticide
input data until 1982, these models were fit on a subset of data from
1982 to present. By adding these more mechanistic drivers to the broad
scale intensification metrics, we are able to see if the response of the
proportion cropland and crop richness remains consistent given
additional drivers that might be correlated to these variables. If the
responses remain consistent, we can be confident that the results from
the long-term data set using proportion cropland and crop richness are
tenable to explain the patterns of bumble bee relative abundance
observed even if the more mechanistic drivers, pasture acreage and
insecticide inputs, are also strong predictors of bumble bee relative
abundance. In addition, we examined variance inflation factors (VIF) for
each model to determine whether variables were problematically
multi-collinear. While a number of other drivers might be directly or
interactively influencing bumble bee relative abundance {[}e.g.,
climate{]}, other studies have shown independent effects of land-use
change and climate (Kerr et al. 2015; Soroye, Newbold, and Kerr 2020),
suggesting it is sufficient to examine agricultural intensification
metrics on their own.

\hypertarget{results}{%
\section{Results}\label{results}}

Bumble bee occurrence records were compiled from GBIF (n=25,271 from
1877 to 2017) and Bumble Bee Watch (n=2,611, from 2007 to 2018) for a
total of 27,882 unique records over 358 of 535 counties (Fig. S1). The
species contained in each dataset were mutually inclusive. We removed
\emph{B. fraternus}, \emph{B. perplexus}, \emph{B. ashtoni}
(\emph{bohemicus}), and \emph{B. variabilis} from our analyses as these
species lacked sufficient records to make meaningful temporal
interpretations of changes in relative abundance and county occupancy.

\hypertarget{changes-in-species-richness}{%
\subsubsection{Changes in species
richness}\label{changes-in-species-richness}}

Rarefied species richness estimates for each temporal bin show estimated
species richness declining 20\% from 15 species in 1824-1925 to 12
species presently (Fig. 1) - a significant negative trend (t1,13 =
6.084, p = 0.0283). All fifteen temporal bin species accumulation curves
rapidly reached an asymptote, indicating that sample sizes with this bin
size were sufficient to capture the diversity of communities within each
bin (Fig. S2). A sharp drop in estimated species richness occurred
between 1952 and 1959, followed by a slight rebound in the next 50
years. The pattern of decreasing species richness was maintained at two
other binning levels, 5 and 8 bins (Fig. S3)

\hypertarget{changes-in-county-occupancy}{%
\subsubsection{Changes in county
occupancy}\label{changes-in-county-occupancy}}

Overall, most species increased in county occupancy over time (Fig. S4).
However, some species of conservation concern showed either a decrease
(\emph{B. terricola}, and \emph{B. pensylvanicus}), or no change
(\emph{B. affinis}) in county occupancy over time despite the increase
in sampling effort that has occurred over the last few decades.

\hypertarget{changes-in-agricultural-intensification}{%
\subsubsection{Changes in agricultural
intensification}\label{changes-in-agricultural-intensification}}

The areal extent of cropland peaked in our region in 1950 (average of
45\% ± 22\% of county area). Since then, it has decreased almost 10\% to
an average of 34\% ± 20\% (Fig. 3A,B). Intensive agriculture has
remained relatively sparse in the north of our study region, while the
highest intensity occurs in the ``corn belt'' that stretches through
southern Minnesota, Iowa, southern Wisconsin, central and northern
Illinois, and northern Indiana. Throughout the last century, the number
of agricultural crops crown has decreased (Fig. 3C,D). Of the 18 crops
for which we compiled data, an average of 12 ± 1 were grown from 1880 -
1950. Since 1950, this number has declined 50\%, with counties today
growing on average 6 ± 1 crops.

The more fine-scale agricultural intensification variables we measured
showed diverging patterns, with pasture acreage declining 90\% from 1982
to present from an average of 3\% of county area in 1982 to 0.3\% in
2012, while the total county area treated with insecticides increased
54\% from an average of 11\% of county area in 1982 to 17\% in 2012.
(Fig. S5). These changes occurred over similar spatial extents,
primarily concentrated in the corn-belt counties throughout the middle
of the study region.

\hypertarget{patterns-of-bumble-bee-relative-abundance-are-related-to-agricultural-intensity}{%
\subsubsection{Patterns of bumble bee relative abundance are related to
agricultural
intensity}\label{patterns-of-bumble-bee-relative-abundance-are-related-to-agricultural-intensity}}

Patterns in bumble bee relative abundance showed strong associations
with trends in agricultural intensification metrics. Overall, we found
that 7 of 13 species decreased in relative abundance, while 3 remained
stable and 3 increased (Fig. 4, Table S1). These estimates of temporal
change in relative abundance modeled concurrently with metrics of
agricultural intensity aligned well with changes estimated from
generalized linear models fitted with time period as the sole predictor
(Appendix 3). Species increasing in relative abundance over time tended
to be positively associated with increases in proportion of agriculture
as well as crop richness (Fig. 4B,C). Among species estimated to be in
decline, county-level proportion of agriculture was negatively
associated with relative abundance in all but two species (Fig. 4B). The
responses of species in decline to crop richness was generally positive,
with a majority increasing in relative abundance in accordance with crop
richness (Fig. 4C).

When we used county-level agricultural statics to predict the
probability of species occurrence within our study region from
1870-2012, we see a marked expansion of common species in space (e.g.,
\emph{B. impatiens}, \emph{B. bimaculatus}) whereas probability of
occupancy decreases across the region for species of conservation
concern (e.g., \emph{B. terricola}, \emph{B. affinis}, \emph{B.
pensylvanicus}, Fig. 3E-H, Fig. S6).

We found that models that included more detailed agricultural metrics,
available from 1982 to present, were clearly associated with patterns of
relative abundance (Fig 6,D,E, Table S3). Pasture acreage had a
consistently positive association on relative abundance (Fig. 6D),
although the acres of pasture per county has markedly decreased over the
last century (Fig. S4B). Proportion of county area treated with
insecticides had a consistently negative effect on species relative
abundance, except for 3 species that exhibited no effect (Fig. 6E,
\emph{B. auricomus}, \emph{B. impatiens}, and \emph{B. pensylvanicus})
and 3 species which were positively associated with increased
insecticide application (\emph{B. affinis}, \emph{B. rufocinctus} and
\emph{B. vagans}). One species of conservation concern, \emph{B.
terricola}, had the strongest negative relationship with proportion of
county treated with insecticides.

\hypertarget{discussion}{%
\section{Discussion}\label{discussion}}

Using bumble bee observations recorded over 140 years across 6
north-central US states and an unprecedented dataset on the historical
patterns of agricultural land use and practices (extent of cropland,
diversity of crops grown, extent of pasture, pesticide use), we explored
the hypothesis that agricultural intensification is associated with
bumble bee declines. Our results support declines of a number of bumble
bee species throughout the region and suggest that increases in
agricultural intensification are associated with these declines.

Human alteration of terrestrial landscapes via habitat loss is cited as
a principle cause of biodiversity declines worldwide (Foley et al. 2005,
2011; Klein et al. 2007; Tilman et al. 2011; Tscharntke et al. 2012).
Our study joins a number of taxa-specific studies that highlight the
threat of modern agricultural practices to biodiversity (e.g., birds
Benton et al. 2002; Robinson and Sutherland 2002; Meehan, Hurlbert, and
Gratton 2010). Recent documented declines in insect populations have
speculated that habitat loss and intensification of agriculture are
likely key contributors to these patterns (Hallmann et al. 2017; Seibold
et al. 2019), though studies explicitly tying long-term trends in
widespread agricultural intensification to bumble bee declines have been
lacking.

Of the 13 species in this study, 3 were found to increase over the study
dataset (1870-2017), 3 remained at similar levels, and 7 were found to
have declined in relative abundance. The declining species in our study
match those found to be in decline elsewhere, including individual state
analyses within our study region (Grixti et al. 2009; Wood et al. 2019),
the US East coast (Richardson et al. 2018; Jacobson et al. 2018), Canada
(Colla and Packer 2008), and North America, generally (Cameron et al.
2011; Colla et al. 2012). Those showing the most concerning trends
include \emph{B. terricola} and \emph{B. pensylvanicus}, both of which
significantly decreased in relative abundance as well as showed distinct
decreases in the predicted extent of their range over time.

With few exceptions, species whose relative abundance was found to be
increasing or stable had either positive or neutral relationships with
agricultural intensity metrics, while those in decline were negatively
associated with increased agricultural intensity. However, agricultural
intensification includes several aspects. While the areal extent of
agriculture (proportion of cropland in county) was negatively associated
with several species, other species seem to thrive in high proportion of
agriculture including \emph{B. pensylvanicus}, a species of conservation
concern. Moreover, bumble bee species richness was greatest during the
period of greatest areal agricultural extent (1900-1930). On the other
hand, the land-use type (e.g., pasture) of and increased diversity of
crops generally have positive effects on bumble bees relative abundance,
while inputs such as pesticides have few positive effects. This suggests
that the extent of agriculture is not the sole putative driver of bumble
bee decline, but rather the manner in which agriculture is practiced
matters greatly. Landscapes with a high areal extent of agriculture may
still support a stable and diverse bumble bee community provided there
is a high diversity of crops and open habitat such as pasture as well as
limited insecticide use.

The mechanisms underlying the observed relationships between bumble bees
and proportion cropland and crop diversity were not tested in this
study. However, we propose that reductions in suitable nesting and
foraging habitat associated with agricultural intensification, along
with the direct/indirect lethality of insecticides, are likely important
underlying factors. Increases in agricultural land have been implicated
in the loss of diverse floral habitat such as tall-grass prairies (Smith
1998; Brown and Schulte 2011), as well as large-scale reductions in
bumble bee forage plants (Carvell et al. 2006; Scheper et al. 2014). A
shift from the diverse cropping systems of the early to mid 1900's to
largely monocultural systems in recent years has undoubtedly altered
total floral resource availability and temporal continuity (Schellhorn,
Gagic, and Bommarco 2015) and has been shown to negatively impact bumble
bee colony development (Hass et al. 2018). While some mass-flowering
crops might benefit bumble bee colony growth (Westphal,
Steffan-Dewenter, and Tscharntke 2009), they may also favor species that
can tolerate highly variable temporal resources {[}Schmid-Hempel and
Schmid-Hempel (1998); Hemberger et al.~in revision{]} and provide little
dietary diversity for foraging bees. Such a limit on pollen and nectar
diversity may have adverse consequences for bumble bees (Vaudo et al.
2015), especially those with limited diet breadth (Kleijn and Raemakers
2008; Wood et al. 2019). Additionally, landscapes that restrict floral
resource availability during key time periods might limit bumble bee
colony growth (Williams, Regetz, and Kremen 2012) and subsequent queen
production (Crone, Williams, and Letters 2016), thereby reducing
population stability over time.

The largely positive response of relative abundance to crop richness
suggests that increasing agricultural landscape heterogeneity could have
benefits for both common and declining bumble bee species. Increasing
landscape heterogeneity is known to benefit a number of ecosystem
services (Fahrig et al. 2011; Turner, Donato, and Romme 2012) and
biodiversity (Benton, Vickery, and Wilson 2003; Sirami et al. 2019),
even with relatively simple changes in cropland heterogeneity. In fact,
a recent study suggests that adjustments to agronomic practices,
specifically decreasing field size, could have larger positive effects
on farmland biodiversity than increasing the amount of semi-natural
habitat in the landscape (Sirami et al. 2019). Moreover, decreasing
field sizes may be easier to implement and incentivize through
traditional farm policy frameworks. As natural habitats are becoming
increasingly rare, alterations that we can make to our land use
practices (e.g., Landis 2017) will be the key toward building
multi-functional agroecosystems that both provide food and fiber for
humans, as well as conserve essential ecosystem service providing
organisms and biodiversity writ large.

The proportion of pasture in a county was positively associated with
bumble bee relative abundance for all but three species (which showed no
relationship). Pastures and grasslands are often a source of floral
habitat for foraging bumble bees both in the US and in Europe (Carvell
et al. 2006) as these open areas often contain plants in the family
Fabaceae, a preferred source of bumble bee forage (Goulson et al. 2005)
especially in several Midwestern bumble bees species (Fye and Medler
1954). While pastures and fabaceous cover crops were used heavily in the
early 20th century to return nitrogen to soils, technological
advancements following the second world war (namely artificial,
nitrogenous fertilizers) led quickly to the decline in these crops
(Williams 1986; Williams and Osborne 2009).

Generally, our analysis of the dataset with insecticide data
(1982-present) showed that increases in the acreage of cropland treated
with insecticides tended to be associated with decreased relative
abundance of all but three species. There is clear evidence that
exposure to insecticides is detrimental to bumble bee foraging (Feltham,
Park, and Goulson 2014), colony development (Whitehorn et al. 2012;
Crall et al. 2018), and may also facilitate increased pathogen
prevalence (McArt et al. 2017). Our analysis provides additional
evidence of the detrimental effect of insecticide use on bumble bees
over a large spatial and temporal scale.

Interestingly, in this dataset we found no evidence of decline in the
relative abundance of the federally endangered \emph{B. affinis} across
our study region. This result stands in contrast to other findings by
almost all other longitudinal studies in which \emph{B. affinis} is
considered (Grixti et al. 2009; Richardson et al. 2018). However, when
examining patterns of \emph{B. affinis} decline across its entire range,
there is clear evidence for extensive range reductions and local
extirpations {[}Cameron et al. (2011); Colla2012{]}. Indeed, the subset
of the range observed in our study has the most extensive modern records
of \emph{B. affinis}, suggesting that our observed regional patterns may
differ from range-wide population trends.

Museum records are an invaluable source of data for studies examining
changes in species range and long-term population trends (Bartomeus et
al. 2019), as well as understanding the ecological drivers that underpin
changes in species assemblages over time (Wood et al. 2019). A strength
of leveraging long-term museum records, including our approach, includes
the ability to examine spatial and temporal scales that are not amenable
to experimental methods (Meehan et al. 2011). However, care must be
taken to account for biases associated with historical collections. For
example, museum collection records do not necessarily adhere to standard
ecological practices of ensuring random, equal-effort samples over
adequate space and time (Richardson et al. 2018). To account for known
issues, we employed a variety of techniques including subsampling and
sensitivity analyses to determine whether our chosen methods skewed
results (for a thorough analysis of techniques to leverage historical
collections, see Bartomeus et al. 2019). Irrespective of the approaches
used, we consistently identified declines in over half of study species
with clear associations to metrics of agricultural intensity. While
there is much evidence for the consequences of agricultural
intensification on bumble bee communities, other variables may also be
driving shifts. For example, climate change has reduced bumble bee
ranges in both North America and Europe (Soroye, Newbold, and Kerr
2020). In their analysis, Soroye, Newbold, and Kerr (2020) found that
climate and land-use changes impact bumble bee occupancy independently.
That is to say that while climate may be driving large-scale range
shifts, land-use still has strong, negative effects on bumble bee
occupancy -- an effect that we show strong evidence for here.

It is important to place the observed changes in relative abundance and
county occupancy into a larger context. Examining changes from 1870 to
present day may not accurately capture the pre-European settlement
baseline of the study species examined. By 1870, many counties were
already extensively altered by both agriculture and deforestation
(Rhemtulla, Mladenoff, and Clayton 2007). In some cases (e.g.,
Michigan), these alterations may have provided more floral resources by
creating open habitat, resulting in the increase of some bumble bee
species (Wood et al. 2019). In shifting from the agronomic practices of
the early 20th century to the more industrialized, intensive practices
of the later 20th century, the suitability for those previously favored
bumble bees decreased paving way for the rise of others. The lack of
true historical baselines does not mean that the declines described in
this study are normal or unworthy of immediate conservation action. The
scale and intensity of agricultural intensification over the last 60
years is unprecedented and the risks to bumble bees clear. The dualism
of bumble bee species responses to agricultural intensification both
historically and contemporarily suggests that species habitat
requirements are variable - something we must explicitly consider in
planning bumble bee conservation strategies.

\hypertarget{conclusions}{%
\subsubsection{Conclusions}\label{conclusions}}

Modern experimental evidence has provided evidence of the effect of
agricultural intensification on bumble bees. By relating changes in
historical, county-level agricultural intensity to changes in the
relative abundance of bumble bee species across the upper Midwest, we
provide historical context of the extent and duration of declines,
further solidifying the evidence that agricultural intensification is a
key driver of shifts in bumble bee community composition and abundance
over the last century. Over the last 140 years, agricultural
intensification -- specifically how agriculture is practiced rather than
simply the amount -- has elicited strong, selective pressures
inadvertently choosing winner and loser species in bumble bee
communities. Our work concurs with an existing body of evidence,
including recent, catastrophic declines documented in insect and
arthropod abundance in agriculturally dominated landscapes. There is
still a great need for additional experimentation to understand how land
use drivers might interact with other known causes of bumble bee
decline, namely climate change and pathogen and disease prevalence. The
combination of our results leveraging museum records and historical data
along with a growing body of experimental evidence suggests that changes
to our agricultural practice and policies are required in order to limit
additional declines of bumble bees in our agricultural landscapes.

\hypertarget{acknowledgements}{%
\section{Acknowledgements}\label{acknowledgements}}

We would like to thank Christelle Guédot, John Orrock, and Russ Groves
for the valuable feedback that improved this manuscript. Many thanks to
Rich Hatfield and the Xerces Society for Invertebrate Conservation for
providing Bumble Bee Watch data. Data and code for all analyses,
figures, and the manuscript will be made publicly available upon
publication at \url{https://github.com/jhemberger}. \clearpage

\newpage

\hypertarget{figures-and-tables}{%
\section{Figures and Tables}\label{figures-and-tables}}

\includegraphics[width=0.6\textwidth,height=\textheight]{../ms_figs/fig1.png}

\textbf{Figure 1:} Trend in rarified bumble bee species richness from
1877-present. Each point represents a date range that is standardized to
contain an approximately equal number of bumble bee records, and it is
plotted at the midpoint year of the date range. Error bars are 95\%
confidence intervals. The fitted line is a linear model predicting
estimated species richness as a function of temporal bin order using the
midpoint of the temporal bin as the predictor. Carpet plot represents
temporal collection year for all records from 1877 to present.

\clearpage

\newpage

\includegraphics[width=1\textwidth,height=\textheight]{../ms_figs/fig2.png}

\textbf{Figure 2:} Patterns of agricultural intensification in two
metrics: (A) proportion of county under cultivation and (C) number of
crops grown per county. Inset graphs (B,D) depict general trend of these
variables for each state in the study area as modeled by a Loess curve.
Predicted probability of occurrence for two example species, \emph{B.
impatiens} (E,F) and \emph{B. terricola} (G,H) in each county given crop
diversity and proportion cropland in each county. Darker colors denote
smaller probability of occurrence, while lighter colors indicate larger
probability of occurrence. Red points on each map are randomly selected
occurrence records with a maximum of 50 plotted per year. (F,H) Fitted
temporal trend of relative abundance when modeled with agricultural
intensity metrics. Solid red lines denote statistically clear trend in
relative abundance over time (p \textless{} 0.05), while dashed lines
denote statistically unclear trends (p \textgreater{} 0.05). Points on
temporal trend are slightly jittered for visibility. Color palettes
derived using quantile binning. Values above maps are mean ± SEM for
each response.

\clearpage

\newpage

\includegraphics[width=1\textwidth,height=\textheight]{../ms_figs/fig3.png}

\textbf{Figure 3:} Fitted model coefficients (± 95\% confidence
interval) for each study species for (A) temporal trend of the species,
(B) proportion of county in cropland, and (C) the number of crops grown
in a county. Models were fitted using the entire dataset from
1870-present. Point colors indicate species relative abundance trend
from 1870-present: yellow points denote species whose relative abundance
declined, gray points denote no change in relative abundance, and black
points denote increases in relative abundance.

\clearpage

\newpage

\includegraphics[width=1\textwidth,height=\textheight]{../ms_figs/fig5.png}

\textbf{Figure 4:} Fitted model coefficients (± 95\% confidence
interval) for each study species for (A) temporal trend of the species,
(B) proportion of county in cropland, and (C) the number of crops grown
in a county, (D) acres of pasture grown in a county, and (E) acres
treated with insecticides. Models were fitted using data from
1982-present (only years for which pesticide and pasture acreage are
available across all counties). Point colors indicate species relative
abundance trend from 1982-present: yellow points denote species whose
relative abundance declined, gray points denote no change in relative
abundance, and black points denote increases in relative abundance.

\clearpage

\newpage

\hypertarget{supplementary-material}{%
\section{Supplementary Material}\label{supplementary-material}}

\hypertarget{appendix-1}{%
\subsubsection{Appendix 1}\label{appendix-1}}

\textbf{Analysis filtering data to unique \emph{species x location x
collector x date} combinations}

\clearpage

\newpage

\hypertarget{appendix-2}{%
\subsubsection{Appendix 2}\label{appendix-2}}

\textbf{Analysis restricting to records within ± 2 years of agricultural
census}

\newpage

\hypertarget{appendix-3}{%
\subsubsection{Appendix 3}\label{appendix-3}}

\textbf{Raw changes in species relative abundance over time}

For each species, we calculated relative abundance for each temporal bin
as the number of records of a given species divided by the total number
of records for all species in that temporal bin. We then tested whether
there was a statistically clear change in relative abundance over time
by fitting generalized linear models with binomial error distributions
for each species using time (i.e., temporal bin) as the sole predictor
with each point weighted by the number of records (similar to Bartomeus
et al. 2013).

Of the 13 species included in our simple analysis of change in relative
abundance over time, 6 saw statistically clear reductions in relative
abundance: \emph{B. borealis}, \emph{B. fervidus}, \emph{B.
pensylvanicus}, \emph{B. rufocinctus}, \emph{B. terricola}, \emph{B.
vagans} (Table S1). Similar to the patterns of estimated species
richness, reductions in the relative abundance of these declining
species seem to begin in the middle of the 20th century. There were 7
species which increased in relative abundance over time period covered
by this dataset: \emph{B. affinis}, \emph{B. auricomus}, \emph{B.
bimaculatus}, \emph{B. citrinus}, \emph{B. griseocollis}, \emph{B.
impatiens}, and \emph{B. ternarius}.

\textbf{Table SA3.1:} Relative abundance estimates for each species
across 8 time bins, sized in order to contain approximately equal
numbers of bumble bee records per bin. Coefficient estimates and
p-values derived from models fit to estimate trends in relative
abundance over time, where relative abundance of each species is
predicted by time bin fit using a generalized linear model, with N
records representing the number of records for a given species.

\includegraphics[width=1\textwidth,height=\textheight]{../ms_figs/table1.png}

\clearpage

\textbf{Supplementary Table 1:} Generalized linear model results for
each study species including the sample size (number of counties for
which relative abundance is calculated), model term, scaled coefficient
estimate, 95\% confidence interval and p-value. Each county is weighted
by the total number of bumble bee records for a given time point
(agricultural census year).

\includegraphics[width=0.6\textwidth,height=\textheight]{../ms_figs/tables1.png}

\clearpage

\textbf{Supplementary Table 3:} Generalized linear model results for
each study species including the sample size (number of counties for
which relative abundance is calculated), model term, scaled coefficient
estimate, 95\% confidence interval and p-value. Each county is weighted
by the total number of bumble bee records for a given time point
(agricultural census year).

\includegraphics[width=0.6\textwidth,height=\textheight]{../ms_figs/tables2.png}

\clearpage

\newpage

\includegraphics[width=0.8\textwidth,height=\textheight]{../ms_figs/fig_s2.png}

\textbf{Supplementary Figure 1:} The location of bumble bee records for
each US Census of Agriculture year considered in our analysis. The
number of records is noted below the year. Points are partially
transparent and jittered slightly for visibility.

\clearpage

\newpage

\includegraphics[width=0.6\textwidth,height=\textheight]{../ms_figs/fig_s3.png}

\textbf{Supplementary Figure 2:} Sample-based species accumulation
curves for each of 15 temporal bins (different colors) from which
estimated species richness was calculated (Fig. 1). Each accumulation
curve is constructed for temporal bins with a roughly equal number of
records.

\clearpage

\newpage

\includegraphics[width=0.6\textwidth,height=\textheight]{../ms_figs/fig_s1.png}

\textbf{Supplementary Figure 3:} Trend in rarified bumble bee species
richness from 1877-present for alternative numbers of temporal bins,
each representing approximately equal numbers of bumble bee records: (A)
5 temporal bins and (B) 8 temporal bins. Each point is plotted at the
midpoint of the temporal bin date range. Error bars are 95\% confidence
intervals. The fitted line is a linear model predicting estimated
species richness as a function of temporal bin order using the midpoint
year of the temporal bin as the predictor. Carpet plot represents
temporal collection year for all records from 1877 to present.

\clearpage

\newpage

\includegraphics[width=1\textwidth,height=\textheight]{../ms_figs/fig6.png}

\textbf{Supplementary Figure 4:} (A) Fitted model coefficients for each
species for change in county occupancy over time. (B) For each species,
plotted temporal trends of county occupancy over time with fitted GLM
models with 95\% confidence interval. Point colors indicate species
relative abundance trend from 1870-present: red points denote species
whose relative abundance declined, black points denote no change in
relative abundance, and blue points denote increases in relative
abundance.

\clearpage

\newpage

\includegraphics[width=1\textwidth,height=\textheight]{../ms_figs/fig_s4.png}

\textbf{Supplementary Figure 5:} Patterns of agricultural
intensification in two additional metrics from 1982-2012: (a) proportion
of county area in pasture and (c) proportion of county area treated with
insecticides. Color palettes derived using quantile binning. Inset
graphs (b,d) depict general trend of these two variables for each state
in the study area as modeled by a Loess curve. Values beneath years are
mean proportion ± SEM.

\clearpage

\newpage

\includegraphics[width=1\textwidth,height=\textheight]{../ms_figs/fig4a.png}

\clearpage

\newpage

\includegraphics[width=1\textwidth,height=\textheight]{../ms_figs/fig4b.png}

\textbf{Supplementary Figure 6:} Fitted temporal trend of relative
abundance when modeled with agricultural intensity metrics along with
the predicted probability of occurrence for each study species in each
county given crop diversity and proportion cropland in each county.
Darker colors denote smaller probability of occurrence, while lighter
colors indicate larger probability of occurrence. Solid red lines denote
statistically clear trend in relative abundance over time (p \textless{}
0.05), while dashed lines denote statistically unclear trends (p
\textgreater{} 0.05). Points on temporal trend are slightly jittered for
visibility. Red points on each map are randomly selected occurrence
records with a maximum of 50 plotted per year.

\clearpage

\hypertarget{references}{%
\section*{References}\label{references}}
\addcontentsline{toc}{section}{References}

\hypertarget{refs}{}
\leavevmode\hypertarget{ref-Bartomeus2013}{}%
Bartomeus, Ignasi, Mia G. Park, Jason Gibbs, Bryan N. Danforth, Alan N.
Lakso, and Rachael Winfree. 2013. ``Biodiversity ensures
plant-pollinator phenological synchrony against climate change.''
\emph{Ecol. Lett.} 16 (11): 1331--8.
\url{https://doi.org/10.1111/ele.12170}.

\leavevmode\hypertarget{ref-Bartomeus2019}{}%
Bartomeus, I., J. R. Stavert, D. Ward, and O. Aguado. 2019. ``Historical
collections as a tool for assessing the global pollination crisis.''
\emph{Philos. Trans. R. Soc. B Biol. Sci.} 374 (1763): 1--9.
\url{https://doi.org/10.1098/rstb.2017.0389}.

\leavevmode\hypertarget{ref-Benton2002}{}%
Benton, Tim G., David M. Bryant, Lorna Cole, and Humphrey Q.P. Crick.
2002. ``Linking agricultural practice to insect and bird populations: A
historical study over three decades.'' \emph{J. Appl. Ecol.} 39 (4):
673--87. \url{https://doi.org/10.1046/j.1365-2664.2002.00745.x}.

\leavevmode\hypertarget{ref-Benton2003}{}%
Benton, Tim G., Juliet A. Vickery, and Jeremy D. Wilson. 2003.
``Farmland biodiversity: Is habitat heterogeneity the key?''
\emph{Trends Ecol. Evol.} 18 (4): 182--88.
\url{https://doi.org/10.1016/S0169-5347(03)00011-9}.

\leavevmode\hypertarget{ref-Biesmeijer2006}{}%
Biesmeijer, J. C. 2006. ``Parallel Declines in Pollinators and
Insect-Pollinated Plants in Britain and the Netherlands.'' \emph{Science
(80-. ).} 313 (5785): 351--54.
\url{https://doi.org/10.1126/science.1127863}.

\leavevmode\hypertarget{ref-Brown2011}{}%
Brown, Paul W., and Lisa A. Schulte. 2011. ``Agricultural landscape
change (1937-2002) in three townships in Iowa, USA.'' \emph{Landsc.
Urban Plan.} 100 (3): 202--12.
\url{https://doi.org/10.1016/j.landurbplan.2010.12.007}.

\leavevmode\hypertarget{ref-Cameron2011}{}%
Cameron, Sydney A, Jeffrey D Lozier, James P Strange, Jonathan B Koch,
Nils Cordes, Leellen F Solter, Terry L Griswold, and Gene E Robinson.
2011. ``Patterns of widespread decline in North American bumble bees.''
\emph{Proc. Natl. Acad. Sci. U. S. A.} 108 (2): 662--67.
\url{https://doi.org/10.1073/pnas.1014743108}.

\leavevmode\hypertarget{ref-Cane2017}{}%
Cane, James H., and Vincent J. Tepedino. 2017. ``Gauging the Effect of
Honey Bee Pollen Collection on Native Bee Communities.'' \emph{Conserv.
Lett.} 10 (2): 205--10. \url{https://doi.org/10.1111/conl.12263}.

\leavevmode\hypertarget{ref-Carvell2006b}{}%
Carvell, Claire, David B. Roy, Simon M. Smart, Richard F. Pywell, Chris
D. Preston, and Dave Goulson. 2006. ``Declines in forage availability
for bumblebees at a national scale.'' \emph{Biol. Conserv.} 132 (4):
481--89. \url{https://doi.org/10.1016/j.biocon.2006.05.008}.

\leavevmode\hypertarget{ref-Colla2012}{}%
Colla, Sheila R., Fawziah Gadallah, Leif Richardson, David Wagner, and
Lawrence Gall. 2012. ``Assessing declines of North American bumble bees
(Bombus spp.) using museum specimens.'' \emph{Biodivers. Conserv.} 21
(14): 3585--95. \url{https://doi.org/10.1007/s10531-012-0383-2}.

\leavevmode\hypertarget{ref-Colla2008}{}%
Colla, Sheila R., and Laurence Packer. 2008. ``Evidence for decline in
eastern North American bumblebees (Hymenoptera: Apidae), with special
focus on Bombus affinis Cresson.'' \emph{Biodivers. Conserv.}
\url{https://doi.org/10.1007/s10531-008-9340-5}.

\leavevmode\hypertarget{ref-Crall2018}{}%
Crall, James D, Callin M Switzer, Robert L Oppenheimer, Ashlee N. Ford
Versypt, Biswadip Dey, Andrea Brown, Mackay Eyster, et al. 2018.
``Neonicotinoid exposure disrupts bumblebee nest behavior, social
networks, and thermoregulation.'' \emph{Science (80-. ).} 362 (6415):
683--86. \url{https://doi.org/10.1126/science.aat1598}.

\leavevmode\hypertarget{ref-Crone2016}{}%
Crone, Elizabeth E., Neal M. Williams, and Ecology Letters. 2016.
``Bumble bee colony dynamics: Quantifying the importance of land use and
floral resources for colony growth and queen production.'' Edited by
Rebecca Irwin. \emph{Ecol. Lett.} 19 (4): 460--68.
\url{https://doi.org/10.1111/ele.12581}.

\leavevmode\hypertarget{ref-Cusser2018}{}%
Cusser, Sarah, John L. Neff, and Shalene Jha. 2018. ``Land-use history
drives contemporary pollinator community similarity.'' \emph{Landsc.
Ecol.} 33 (8): 1335--51.
\url{https://doi.org/10.1007/s10980-018-0668-2}.

\leavevmode\hypertarget{ref-Fahrig2011b}{}%
Fahrig, Lenore, Jacques Baudry, Lluís Brotons, Françoise G Burel, Thomas
O Crist, Robert J Fuller, Clelia Sirami, Gavin M Siriwardena, and
Jean-Louis Martin. 2011. ``Functional landscape heterogeneity and animal
biodiversity in agricultural landscapes.'' \emph{Ecol. Lett.} 14 (2):
101--12. \url{https://doi.org/10.1111/j.1461-0248.2010.01559.x}.

\leavevmode\hypertarget{ref-Feltham2014}{}%
Feltham, Hannah, Kirsty Park, and Dave Goulson. 2014. ``Field realistic
doses of pesticide imidacloprid reduce bumblebee pollen foraging
efficiency.'' \emph{Ecotoxicology} 23 (3): 317--23.
\url{https://doi.org/10.1007/s10646-014-1189-7}.

\leavevmode\hypertarget{ref-Foley2005a}{}%
Foley, Jonathan A., Ruth DeFries, Gregory P. Asner, Carol Barford,
Gordon Bonan, Stephen R. Carpenter, F. Stuart Chapin, et al. 2005.
``Global consequences of land use.'' \emph{Science (80-. ).} 309 (5734):
570--74. \url{https://doi.org/10.1126/science.1111772}.

\leavevmode\hypertarget{ref-Foley2011b}{}%
Foley, Jonathan A., Navin Ramankutty, Kate A. Brauman, Emily S. Cassidy,
James S. Gerber, Matt Johnston, Nathaniel D. Mueller, et al. 2011.
``Solutions for a cultivated planet.'' \emph{Nature} 478 (7369):
337--42. \url{https://doi.org/10.1038/nature10452}.

\leavevmode\hypertarget{ref-Fye1954a}{}%
Fye, R E, and J T Medler. 1954. ``Spring emergence and floral hosts of
Wisconsin bumblebees.''

\leavevmode\hypertarget{ref-Giles2006}{}%
Giles, Valerie, and JOHN Ascher. 2006. ``A survey of the bees of the
Black Rock Forest preserve, New York (Hymenoptera: Apoidea).'' \emph{J.
Hymenopt. Res.} 15 (2): 208--31.

\leavevmode\hypertarget{ref-Goulson2015c}{}%
Goulson, Dave, Elizabeth Nicholls, C. Botias, Ellen L. Rotheray,
Cristina Botías, and Ellen L. Rotheray. 2015. ``Bee declines driven by
combined stress from parasites, pesticides, and lack of flowers.''
\emph{Science (80-. ).}, no. February: 1--16.
\url{https://doi.org/10.1126/science.1255957}.

\leavevmode\hypertarget{ref-Goulson2005b}{}%
Goulson, D., M. E. Hanley, B. Darvill, J. S. Ellis, and M. E. Knight.
2005. ``Causes of rarity in bumblebees.'' \emph{Biol. Conserv.} 122 (1):
1--8. \url{https://doi.org/10.1016/j.biocon.2004.06.017}.

\leavevmode\hypertarget{ref-Grixti2009}{}%
Grixti, Jennifer C., Lisa T. Wong, Sydney a. Cameron, and Colin Favret.
2009. ``Decline of bumble bees (Bombus) in the North American Midwest.''
\emph{Biol. Conserv.} 142 (1): 75--84.
\url{https://doi.org/10.1016/j.biocon.2008.09.027}.

\leavevmode\hypertarget{ref-Hallmann2017}{}%
Hallmann, Caspar A., Martin Sorg, Eelke Jongejans, Henk Siepel, Nick
Hofland, Heinz Schwan, Werner Stenmans, et al. 2017. ``More than 75
percent decline over 27 years in total flying insect biomass in
protected areas.'' \emph{PLoS One} 12 (10): e0185809.
\url{https://doi.org/10.1371/journal.pone.0185809}.

\leavevmode\hypertarget{ref-Hass2018a}{}%
Hass, Annika Louise, Lara Brachmann, Péter Batáry, Yann Clough, Hermann
Behling, and Teja Tscharntke. 2018. ``Maize-dominated landscapes reduce
bumblebee colony growth through pollen diversity loss.'' \emph{J. Appl.
Ecol.}, no. September: 1--11.
\url{https://doi.org/10.1111/1365-2664.13296}.

\leavevmode\hypertarget{ref-Hsieh2016}{}%
Hsieh, T. C., K. H. Ma, and Anne Chao. 2016. ``iNEXT: an R package for
rarefaction and extrapolation of species diversity (Hill numbers).''
\emph{Methods Ecol. Evol.} 7 (12): 1451--6.
\url{https://doi.org/10.1111/2041-210X.12613}.

\leavevmode\hypertarget{ref-Jacobson2018a}{}%
Jacobson, Molly M., Erika M. Tucker, Minna E. Mathiasson, and Sandra M.
Rehan. 2018. ``Decline of bumble bees in northeastern North America,
with special focus on Bombus terricola.'' \emph{Biol. Conserv.} 217
(August 2017): 437--45.
\url{https://doi.org/10.1016/j.biocon.2017.11.026}.

\leavevmode\hypertarget{ref-Kerr2015}{}%
Kerr, Jeremy T, Alana Pindar, Paul Galpern, Laurence Packer, Simon G
Potts, Stuart M Roberts, Pierre Rasmont, et al. 2015. ``Climate change
impacts on bumblebees converge across continents.'' \emph{Science (80-.
).} 349 (6244): 177--80. \url{https://doi.org/10.1126/science.aaa7031}.

\leavevmode\hypertarget{ref-Kleijn2008}{}%
Kleijn, David, and Ivo Raemakers. 2008. ``A retrospective analysis of
pollen host plant use by stable and declining bumble bee species.''
\emph{Ecology} 89 (7): 1811--23.
\url{https://doi.org/10.1890/07-1275.1}.

\leavevmode\hypertarget{ref-Klein2007g}{}%
Klein, Alexandra-Maria, Bernard E Vaissière, James H Cane, Ingolf
Steffan-Dewenter, Saul a Cunningham, Claire Kremen, and Teja Tscharntke.
2007. ``Importance of pollinators in changing landscapes for world
crops.'' \emph{Proc. Biol. Sci.} 274 (1608): 303--13.
\url{https://doi.org/10.1098/rspb.2006.3721}.

\leavevmode\hypertarget{ref-Landis2017}{}%
Landis, Douglas A. 2017. ``Designing agricultural landscapes for
biodiversity-based ecosystem services.'' \emph{Basic Appl. Ecol.} 18:
1--12. \url{https://doi.org/10.1016/j.baae.2016.07.005}.

\leavevmode\hypertarget{ref-Liczner2019}{}%
Liczner, Amanda R., and Sheila R. Colla. 2019. ``A systematic review of
the nesting and overwintering habitat of bumble bees globally.''
\emph{J. Insect Conserv.} 23 (5): 787--801.
\url{https://doi.org/10.1007/s10841-019-00173-7}.

\leavevmode\hypertarget{ref-Mallinger2017a}{}%
Mallinger, Rachel E, Hannah R. Gaines-Day, and Claudio Gratton. 2017.
``Do managed bees have negative effects on wild bees?: A systematic
review of the literature.'' Edited by Nigel E. Raine. \emph{PLoS One} 12
(12): e0189268. \url{https://doi.org/10.1371/journal.pone.0189268}.

\leavevmode\hypertarget{ref-Marshall2017}{}%
Marshall, Leon, Jacobus C. Biesmeijer, Pierre Rasmont, Nicolas J.
Vereecken, Libor Dvorak, Una Fitzpatrick, Frédéric Francis, et al. 2017.
``The interplay of climate and land use change affects the distribution
of EU bumblebees.'' \emph{Glob. Chang. Biol.}, no. July: 1--16.
\url{https://doi.org/10.1111/gcb.13867}.

\leavevmode\hypertarget{ref-McArt2017}{}%
McArt, Scott H., Christine Urbanowicz, Shaun McCoshum, Rebecca E. Irwin,
and Lynn S. Adler. 2017. ``Landscape predictors of pathogen prevalence
and range contractions in US bumblebees.'' \emph{Proc. R. Soc. B Biol.
Sci.} 284 (1867): 20172181.
\url{https://doi.org/10.1098/rspb.2017.2181}.

\leavevmode\hypertarget{ref-Meehan2015}{}%
Meehan, Timothy D., and Claudio Gratton. 2015. ``A consistent positive
association between landscape simplification and insecticide use across
the Midwestern US from 1997 through 2012.'' \emph{Environ. Res. Lett.}
10 (11). \url{https://doi.org/10.1088/1748-9326/10/11/114001}.

\leavevmode\hypertarget{ref-Meehan2010a}{}%
Meehan, Timothy D, Allen H Hurlbert, and Claudio Gratton. 2010. ``Bird
communities in future bioenergy landscapes of the Upper Midwest.''
\emph{Proc. Natl. Acad. Sci. U. S. A.} 107 (43): 18533--8.
\url{https://doi.org/10.1073/pnas.1008475107}.

\leavevmode\hypertarget{ref-Meehan2011}{}%
Meehan, Timothy D, Ben P Werling, Douglas a Landis, and Claudio Gratton.
2011. ``Agricultural landscape simplification and insecticide use in the
Midwestern United States.'' \emph{Proc. Natl. Acad. Sci. U. S. A.} 108
(28): 11500--11505. \url{https://doi.org/10.1073/pnas.1100751108}.

\leavevmode\hypertarget{ref-Morales2013}{}%
Morales, Carolina L., Marina P. Arbetman, Sydney A. Cameron, and Marcelo
A. Aizen. 2013. ``Rapid ecological replacement of a native bumble bee by
invasive species.'' \emph{Front. Ecol. Environ.} 11 (10): 529--34.
\url{https://doi.org/10.1890/120321}.

\leavevmode\hypertarget{ref-Ollerton2011}{}%
Ollerton, Jeff, Rachael Winfree, and Sam Tarrant. 2011. ``How many
flowering plants are pollinated by animals?'' \emph{Oikos} 120 (3):
321--26. \url{https://doi.org/10.1111/j.1600-0706.2010.18644.x}.

\leavevmode\hypertarget{ref-Pearce2006}{}%
Pearce, Jennie L., and Mark S. Boyce. 2006. ``Modelling distribution and
abundance with presence-only data.'' \emph{J. Appl. Ecol.} 43 (3):
405--12. \url{https://doi.org/10.1111/j.1365-2664.2005.01112.x}.

\leavevmode\hypertarget{ref-Rasmont2012}{}%
Rasmont, Pierre, and Stéphanie Iserbyt. 2012. ``The bumblebees scarcity
syndrome: Are heat waves leading to local extinctions of bumblebees
(Hymenoptera: Apidae: Bombus)?'' \emph{Ann. La Soc. Entomol. Fr.} 48
(3-4): 275--80. \url{https://doi.org/10.1080/00379271.2012.10697776}.

\leavevmode\hypertarget{ref-Rhemtulla2007a}{}%
Rhemtulla, Jeanine M., David J. Mladenoff, and Murray K. Clayton. 2007.
``Regional land-cover conversion in the U.S. upper Midwest: Magnitude of
change and limited recovery (1850-1935-1993).'' \emph{Landsc. Ecol.} 22
(SUPPL. 1): 57--75. \url{https://doi.org/10.1007/s10980-007-9117-3}.

\leavevmode\hypertarget{ref-Richardson2018}{}%
Richardson, L.L, K.P McFarland, S Zahendra, and S Hardy. 2018. ``Bumble
bee (\textless{}i\textgreater{}Bombus\textless{}/i\textgreater{})
distribution and diversity in Vermont, USA: A century of change.''
\emph{J. Insect Conserv.} In Review (0): 0.
\url{https://doi.org/10.1007/s10841-018-0113-5}.

\leavevmode\hypertarget{ref-Robinson2002}{}%
Robinson, Robert A., and William J. Sutherland. 2002. ``Post-war changes
in arable farming and biodiversity in Great Britain.'' \emph{J. Appl.
Ecol.} 39 (1): 157--76.
\url{https://doi.org/10.1046/j.1365-2664.2002.00695.x}.

\leavevmode\hypertarget{ref-Ropars2019}{}%
Ropars, Lise, Isabelle Dajoz, Colin Fontaine, Audrey Muratet, and Benoît
Geslin. 2019. ``Wild pollinator activity negatively related to honey bee
colony densities in urban context.'' Edited by Wolfgang Blenau.
\emph{PLoS One} 14 (9): e0222316.
\url{https://doi.org/10.1371/journal.pone.0222316}.

\leavevmode\hypertarget{ref-Schellhorn2015c}{}%
Schellhorn, Nancy A., Vesna Gagic, and Riccardo Bommarco. 2015. ``Time
will tell: Resource continuity bolsters ecosystem services.''
\emph{Trends Ecol. Evol.} 30 (9): 524--30.
\url{https://doi.org/10.1016/j.tree.2015.06.007}.

\leavevmode\hypertarget{ref-Scheper2014}{}%
Scheper, Jeroen, Menno Reemer, Ruud van Kats, Wim A. Ozinga, Giel T. J.
van der Linden, Joop H. J. Schaminée, Henk Siepel, and David Kleijn.
2014. ``Museum specimens reveal loss of pollen host plants as key factor
driving wild bee decline in The Netherlands.'' \emph{Proc. Natl. Acad.
Sci.} 111 (49): 201412973.
\url{https://doi.org/10.1073/pnas.1412973111}.

\leavevmode\hypertarget{ref-Schmid-Hempel1998a}{}%
Schmid-Hempel, Regula, and Paul Schmid-Hempel. 1998. ``Colony
performance and immunocompetence of a social insect, Bombus terrestris,
in poor and variable environments.'' \emph{Funct. Ecol.} 12 (1): 22--30.
\url{https://doi.org/10.1046/j.1365-2435.1998.00153.x}.

\leavevmode\hypertarget{ref-Seibold2019}{}%
Seibold, Sebastian, Martin M Gossner, Nadja K Simons, Nico Blüthgen,
Didem Ambarl, Christian Ammer, Jürgen Bauhus, et al. 2019. ``Arthropod
decline in grasslands and forests is associated with drivers at
landscape level.'' \emph{Nature} 574 (February): 1--34.
\url{https://doi.org/10.1038/s41586-019-1684-3}.

\leavevmode\hypertarget{ref-Sirami2019}{}%
Sirami, Clélia, Nicolas Gross, Aliette Bosem Baillod, Colette Bertrand,
Romain Carrié, Annika Hass, Laura Henckel, et al. 2019. ``Increasing
crop heterogeneity enhances multitrophic diversity across agricultural
regions.'' \emph{Proc. Natl. Acad. Sci.} 116 (33): 201906419.
\url{https://doi.org/10.1073/pnas.1906419116}.

\leavevmode\hypertarget{ref-Sirois-Delisle2018}{}%
Sirois-Delisle, Catherine, and Jeremy T. Kerr. 2018. ``Climate
change-driven range losses among bumblebee species are poised to
accelerate.'' \emph{Sci. Rep.} 8 (1): 14464.
\url{https://doi.org/10.1038/s41598-018-32665-y}.

\leavevmode\hypertarget{ref-Smith1998}{}%
Smith, Daryl D. 1998. ``Iowa prairie: Original extent and loss,
preservation and recovery attempts.'' \emph{J. Iowa Acad. Sci.} 105 (3):
94--108.

\leavevmode\hypertarget{ref-Soroye2020}{}%
Soroye, Peter, Tim Newbold, and Jeremy Kerr. 2020. ``Climate change
contributes to widespread declines among bumble bees across
continents.'' \emph{Science (80-. ).} 367 (6478): 685--88.
\url{https://doi.org/10.1126/science.aax8591}.

\leavevmode\hypertarget{ref-Steffan-Dewenter2005c}{}%
Steffan-Dewenter, Ingolf, Simon G. Potts, Laurence Packer, and Jaboury
Ghazoul. 2005. ``Pollinator diversity and crop pollination services are
at risk {[}3{]} (multiple letters).'' \emph{Trends Ecol. Evol.} 20 (12):
651--53. \url{https://doi.org/10.1016/j.tree.2005.09.004}.

\leavevmode\hypertarget{ref-Tilman2011}{}%
Tilman, David, Christian Balzer, Jason Hill, and Belinda L. Befort.
2011. ``Global food demand and the sustainable intensification of
agriculture.'' \emph{Proc. Natl. Acad. Sci. U. S. A.} 108 (50):
20260--4. \url{https://doi.org/10.1073/pnas.1116437108}.

\leavevmode\hypertarget{ref-Torne-Noguera2016}{}%
Torné-Noguera, Anna, Anselm Rodrigo, Sergio Osorio, and Jordi Bosch.
2016. ``Collateral effects of beekeeping: Impacts on pollen-nectar
resources and wild bee communities.'' \emph{Basic Appl. Ecol.} 17 (3):
199--209. \url{https://doi.org/10.1016/j.baae.2015.11.004}.

\leavevmode\hypertarget{ref-Tscharntke2012}{}%
Tscharntke, Teja, Yann Clough, Thomas C. Wanger, Louise Jackson, Iris
Motzke, Ivette Perfecto, John Vandermeer, and Anthony Whitbread. 2012.
``Global food security, biodiversity conservation and the future of
agricultural intensification.'' \emph{Biol. Conserv.} 151 (1): 53--59.
\url{https://doi.org/10.1016/j.biocon.2012.01.068}.

\leavevmode\hypertarget{ref-Turner2012}{}%
Turner, Monica G., Daniel C. Donato, and William H. Romme. 2012.
``Consequences of spatial heterogeneity for ecosystem services in
changing forest landscapes: priorities for future research.''
\emph{Landsc. Ecol.} 28 (6): 1081--97.
\url{https://doi.org/10.1007/s10980-012-9741-4}.

\leavevmode\hypertarget{ref-Tylianakis2013a}{}%
Tylianakis, Jason M. 2013. ``The global plight of pollinators.''
\emph{Science (80-. ).} 339 (6127): 1532--3.
\url{https://doi.org/10.1126/science.1235464}.

\leavevmode\hypertarget{ref-Vaudo2015}{}%
Vaudo, Anthony D., John F. Tooker, Christina M. Grozinger, and Harland
M. Patch. 2015. ``Bee nutrition and floral resource restoration.''
\emph{Curr. Opin. Insect Sci.} 10: 133--41.
\url{https://doi.org/10.1016/j.cois.2015.05.008}.

\leavevmode\hypertarget{ref-Westphal2009a}{}%
Westphal, C., I. Steffan-Dewenter, and T. Tscharntke. 2009. ``Mass
flowering oilseed rape improves early colony growth but not sexual
reproduction of b1. Westphal, C., Steffan-Dewenter, I. \& Tscharntke, T.
(2009). Mass flowering oilseed rape improves early colony growth but not
sexual reproduction of bumblebees. J.'' \emph{J. Appl. Ecol.} 46 (1):
187--93. \url{https://doi.org/10.1111/j.1365-2664.2008.01580.x}.

\leavevmode\hypertarget{ref-Whitehorn2012}{}%
Whitehorn, Penelope R., Stephanie O'Connor, Felix L. Wackers, and Dave
Goulson. 2012. ``Neonicotinoid Pesticide Reduces Bumble Bee Colony
Growth and Queen Production.'' \emph{Science (80-. ).} 336 (6079):
351--52. \url{https://doi.org/10.1126/science.1215025}.

\leavevmode\hypertarget{ref-Williams2012b}{}%
Williams, Neal M., James Regetz, and Claire Kremen. 2012.
``Landscape-scale resources promote colony growth but not reproductive
performance of bumble bees.'' \emph{Ecology} 93 (5): 1049--58.
\url{https://doi.org/10.1890/11-1006.1}.

\leavevmode\hypertarget{ref-Williams1986}{}%
Williams, Paul H. 1986. ``Environmental change and the distributions of
british bumble bees (bombus latr.).'' \emph{Bee World} 67 (2): 50--61.
\url{https://doi.org/10.1080/0005772X.1986.11098871}.

\leavevmode\hypertarget{ref-Williams2007}{}%
Williams, Paul H., Miguel B. Araújo, and Pierre Rasmont. 2007. ``Can
vulnerability among British bumblebee (Bombus) species be explained by
niche position and breadth?'' \emph{Biol. Conserv.} 138 (3-4): 493--505.
\url{https://doi.org/10.1016/j.biocon.2007.06.001}.

\leavevmode\hypertarget{ref-Williams2009}{}%
Williams, Paul, and Juliet L Osborne. 2009. ``Bumblebee vulnerability
and conservation world-wide.'' \emph{Apidologie} 40: 367--87.

\leavevmode\hypertarget{ref-Wood2019}{}%
Wood, Thomas J., Jason Gibbs, Kelsey K. Graham, and Rufus Isaacs. 2019.
``Narrow pollen diets are associated with declining Midwestern bumble
bee species.'' \emph{Ecology} 0 (0).
\url{https://doi.org/10.1002/ecy.2697}.


\end{document}
